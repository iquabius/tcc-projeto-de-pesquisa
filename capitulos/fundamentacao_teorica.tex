\section{Fundamentação Teórica}
\label{sec:fund-teor}

\subsection{Interfaces Gráficas do Usuário (GUIs)}
\label{sec:guis}

\subsubsection{GUIs em Aplicações Desktop}
\label{sec:guis-desktop}

% - utilizam toolkits

\subsubsection{GUIs em Aplicações Web}
\label{sec:guis-web}

%\subsubsubsection{JavaScript}
%\label{sec:javascript}
%
%\subsubsubsection{DOM \emph{(Document Object Model)}}
%\label{sec:dom}
%
%% \subsubsection{Programação com \emph{Callbacks}}

\subsection{Teoria da Computação}
\label{sec:teoria-da-computacao}

O fenômeno da computação na qual a Ciência da Computação é fundada tem suas
raízes na teoria da computabilidade que estuda os formalismos na qual pode-se
expressar algorítimos e seus limites (Gabbrielli, 2010).

{\tiny
  \begin{description}[noitemsep]
  \item[Tese de Church-Turing] ``Hipótese sobre a natureza de dispositivos
    mecânicos de cálculo, como computadores, e sobre que tipo de algoritmos eles
    podem computar'';
  \item[Máquina de Turing] É um modelo matemático de computação que define uma
    máquina, abstrata, que manipula simbolos em uma tira de fita de acordo com
    uma tabela de regras;
  \item[Cálculo Lambda] ``Sistema formal que estuda funções recursivas
    computáveis, no que se refere a teoria da computabilidade'';
  \end{description}
}

\subsection{Paradigmas de Programação}
\label{sec:prog_paradigms}

A função \emph{doubleNumbers} no Fragmento de Código
\ref{code:fn_double_numbers_imperative} recebe uma lista de números e produz numa nova
lista onde todos os números são multiplicados por 2. Essa implementação utiliza
o estilo \emph{imperativo} de programação.

\begin{listing}[H]
  \centering
  \caption{Dobrando números de uma lista de forma imperativa}
  \inputminted{js}{code/fn_double_numbers_imperative.js}
  \label{code:fn_double_numbers_imperative}
\end{listing}


\subsubsection{Programação Orientada a Objetos}
\label{sec:oop}

{\tiny\begin{description}[noitemsep]
  \item [Design Patterns] \emph{Observer Pattern}
  \item [\emph{Callbacks}]
\end{description}}

\subsubsection{Programação Funcional}
\label{sec:fp}

% "Functional Programming (enabled by lambdas with closure)"
% Contextualização Histórica
% Renascença da Programação Funcional

{\tiny
\begin{description}[noitemsep]
\item[Lambda Calculus]
\item[Programação declarativa]
\item[Aplicação de funções] \emph{Function application}
\item[Avaliações de expressões] \emph{expression evaluation}
\item[Imutabilidade] Possibilita Transparência Referêncial
\item[Efeitos colaterais] \emph{side-effects}
\item[Funções Puras] \emph{Pure Functions}
\item[Funções de primeira class] \emph{first-class functions}
\item[Funções de Ordem Superior] \emph{(Higher-order Functions)}, Funções como
  argumentos, Funções como valores retornados
\item[Primitivas] \emph{map}, \emph{filter}, \emph{fold}, \emph{reduce},
  \emph{scan}, \emph{zip}
\item[Composição de funções] \emph{Function Composition}
\item[Currying]
\end{description}
}

Funções de Primeira Classe \emph{(First Class Functions)}…

% - Expressões lambda / funções anônimas / Closures

\begin{listing}[H]
  \centering
  \caption{Atribuição de funções a variáveis}
  \inputminted{js}{code/fp_first_class_functions.js}
  \label{code:fp_first_class_functions}
\end{listing}

\begin{listing}[H]
  \centering
  \caption{Expressões \emph{lambda}}
  \inputminted{js}{code/fp_lambdas.js}
  \label{code:fp_lambdas}
\end{listing}

Funções de Ordem Superior \emph{(Higher-order Functions)}…


\emph{Funções de Ordem Superior} são funções que aceitam outras funções como
argumento, ou retornam uma função.


Primitivas Básicas de Programação Funcional…


As primitivas, ou operações, \emph{map} e \emph{filter} são funções de ordem superior.

\begin{listing}[H]
  \centering
  \caption{Primitiva \emph{map}}
  \inputminted{js}{code/fp_primitives_map.js}
  \label{code:fp_primitives_map}
\end{listing}

\begin{listing}[H]
  \centering
  \caption{Primitiva \emph{filter}}
  \inputminted{js}{code/fp_primitives_filter.js}
  \label{code:fp_primitives_filter}
\end{listing}


\begin{listing}[H]
  \centering
  \caption{Dobrando números de uma lista de forma declarativa}
  \inputminted{js}{code/fn_double_numbers_declarative.js}
  \label{code:fn_double_numbers_declarative}
\end{listing}


\subsubsection{Programação Reativa}
\label{sec:rp}



\subsubsection{Programação Funcional Reativa (PFR)}
\label{sec:frp}

% "FRP permits the modeling of systems that must respond to input over time in a
% simple and declarative manner." ~ Amsden (2011), Survey on FRP

% "A program in an FRP language generally corresponds quite closely to a
% mathematical model of the system being implemented." ~ Amsden (2011), Survey
% on FRP
%   - Programação Reativa
%     - “[…] is programming with asynchronous data streams” – André Staltz
%   - merge, replay, retry, skip, start, startWith

{\tiny\begin{description}[noitemsep]
  \item [As 10 Primitivas Básicas] \emph{map}, \emph{merge}, \emph{hold},
    \emph{snapshot}, \emph{filter}, \emph{lift}, \emph{never}, \emph{constant},
    \emph{sample}, \emph{switch};
  \item [Combinação de Primitivas]
  \item [Arcabouços \emph{(Frameworks)}] Rx.JS, Bacon.js
\end{description}}

% Ferramentas
%   Bibliotecas & Frameworks
%   Bacon.js
%   Cycle.js → Model-View-Intent
%   Elm → Model-Update-View
%   Rx
%   Meteor

%%% Local Variables:
%%% mode: latex
%%% TeX-master: "../projeto"
%%% End:
