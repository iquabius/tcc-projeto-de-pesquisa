\section{Hipóteses}\label{lhipoteses}


\subsection{Hipótese Básica}

A interatividade intrínseca de interfaces gráficas de
aplicações web lhes dão a característica de serem altamente
orientadas a eventos,
ou seja, precisam reagir a vários comandos do usuário e/ou
mensagens de servidores remotos.
Atualmente o paradigma de \emph{Programação Orientada a
Eventos (EDP -- Event-driven Programming)} é o mais utilizado em
aplicações interativas, no entando muitos o consideram uma
das formas mais complexas para se desenvolver tais aplicações.

Recentemente o paradigma \emph{FRP} tem sido explorado como
uma alternativa promissora, pois permite que aplicações
interativas sejam programadas de forma declarativa em um
nível mais elevado\footnote{
  Em seu ensaio, \citeauthoronline{braithwaite2005whywhy} elucida:
  \textquote[{\citeyear{braithwaite2005whywhy}, tradução nossa}]
    {\textelp{} quando o código se assemelha muito com a forma com
     a qual você explicaria sua solução pelo telefone, costuma-se
     dizer que ele é de \enquote{nível muito alto.}}.
}
de abstração, resultando em código fonte que expressa melhor a
solução implementada.


\subsection{Hipóteses Secundárias}

\begin{itemize}[noitemsep]
  \item A expressividade de um sistema \emph{FRP} pode
        realçar a legibilidade do código.
  \item O paradigma \emph{FRP} pode ser difícil de ser adotado
        devido ao alto nível de abstração.
  \item O processo de \emph{depuração (debugging)} pode ser mais difícil,
        devido a falta de ferramentas adequadas.
\end{itemize}

% Hipóteses
% - PF torna um sistema mais previsível através do gerenciamento do estado
% - PF oferece melhor reuso de código através da composicionalidade
