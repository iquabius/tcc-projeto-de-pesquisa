\section{Hipóteses}\label{lhipoteses}

O \emph{Observer Pattern} -- inerentemente orientado a objetos
e \emph{imperativo} -- utiliza \emph{callbacks}\footnote{
  \emph{Callbacks} são funções registradas para tratar eventos
  ou computações assíncronas, e representão essêncialmente o
  mesmo conceito de \emph{event handlers}, \emph{event
  listeners} ou \emph{observers}.
}
como principal mecanismo de coordenação de eventos em GUIs.
Porém a forma desconcertante com a qual \emph{callbacks}
coordenão eventos faz com que o programa se torne difícil
de compreender e, de modo geral, complexo e de
manutenção laboriosa.

PFR permite que aplicações interativas sejam programadas de forma
\emph{declarativa} em um nível mais elevado\footnote{
  Em seu ensaio, \citeauthoronline{braithwaite2005whywhy} elucida:
  \textquote[{\citeyear{braithwaite2005whywhy}, tradução nossa}]
    {\textelp{} quando o código se assemelha muito com a forma com
     a qual você explicaria sua solução pelo telefone, costuma-se
     dizer que ele é de \enquote{nível muito alto.}}.
}
de abstração, com código fonte que expressa melhor a
solução implementada.
Como resultado, o programa se torna mais compreensível e,
em geral, menos complexo e mais fácil de dar manutenção.
