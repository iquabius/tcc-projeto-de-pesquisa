\section{Hipóteses}\label{lhipoteses}


\subsection{Hipótese Básica}

A interatividade intrínseca de interfaces gráficas de
aplicações web lhes dão a característica de serem altamente
orientadas a eventos,
ou seja, precisam reagir a vários comandos do usuário e/ou
mensagens de servidores remotos.
Atualmente o paradigma de \emph{Programação Orientada a
Eventos (EDP -- Event-driven Programming)} é o mais utilizado em
aplicações interativas, no entando muitos o consideram uma
das formas mais complexas para se desenvolver tais aplicações.

Recentemente o paradigma \emph{FRP} tem sido explorado como
uma alternativa promissora para a programação de aplicações
interativas, devido a maneira declarativa que é oferecida
para se expressar código de coordenação de eventos,
resultando em melhorias nos processos de desenvolvimento,
manutenção e testes de softwares.


\subsection{Hipóteses Secundárias}

\begin{itemize}[noitemsep]
  \item \emph{FRP} fornece um modelo de programação com um
        nível mais elevado de abstração.
  \item A expressividade de um sistema \emph{FRP} pode
        realçar a legibilidade do código.
  \item O paradigma \emph{FRP} pode ser difícil de ser adotado
        devido ao alto nível de abstração.
  \item O processo de \emph{debugging} pode ser mais difícil,
        devido a falta de ferramentas adequadas.
\end{itemize}

% Hipóteses
% - PF torna um sistema mais previsível através do gerenciamento do estado
% - PF oferece melhor reuso de código através da composicionalidade
