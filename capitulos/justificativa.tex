\section{Justificativa}\label{ljustificativa}

Interatividade em páginas web se deu com a
introdução do \emph{JavaScript} nos navegadores,
e desde o advento do \emph{Ajax}\footnote{
  \emph{Ajax (Asynchronous JavaScript + XML)} é uma
  abordagem que abrange várias técnicas e
  tecnologias que permitem a criação de páginas web
  dinâmicas onde a troca de conteúdo entre cliente e
  servidor ocorre de forma assíncrona, sem a necessidade
  de recarregar a página toda \cite{garrett2005ajax}.
},
elas têm se tornado cada vez mais interativas,
dando origem a uma nova gama de aplicações web com interfaces
ricas\footnote{
  Alguns exemplos de aplicações ricas são:
  calendários (\emph{Google Calendar}),
  clientes de email (\emph{Gmail} e \emph{Outlook}),
  \emph{chats} (\emph{Facebook Messenger}),
  editores de texto (\emph{Google Docs}),
  mapas (\emph{Google Maps}, \emph{Waze}) e
  planilhas (\emph{Google Sheets}).
  Leia mais no TCC de \citeonline{kuntze2008aplicacoesricas}:
  \emph{Aplicações Ricas com Ajax}.
} e responsivas que oferecem ao usuário uma experiência
similar as aplicações \emph{mobile} ou \emph{desktop}.
Assim como em qualquer interface gráfica, interfaces web
precisam reagir a vários eventos imprevisíveis do ambiente
externo, provindos tanto dos usuários (e.g. \emph{clicks}
do \emph{mouse}, pressionamento de teclas, gestos multitoque,
etc.) quanto de outros sistemas (e.g. mensagens do servidor,
sinais de sensores, etc.).
Devido a esses aspectos, interfaces gráficas são caracterizadas
como \emph{reativas}.

O \emph{Observer Pattern}, usado na \emph{Programação Orientada
a Objetos (POO)}, é o modelo tradicionalmente predominante na
programação de GUIs.
Inerentemente \emph{imperativo}, ele utiliza \emph{callbacks}
como principal mecanismo de coordenação de eventos.
No entanto, abordagens baseadas em \emph{callbacks} são
consideradas muito complexas \cite{
  edwards2009coherent,
  fischer2007tasks,
  maier2010deprecating,
  reppy1992higher},
devido a forma desconcertante com que \emph{callbacks}
coordenam alterações no estado do programa.
O programa se torna difícil de compreender e dar manutenção,
podendo ainda ser descrito coloquialmente como \enquote{\emph{Callback Hell}}
\cite[p.~2]{edwards2009coherent}, como é costumeiro na literatura.

Vale ressaltar que de acordo com \citeonline{moseley06out}
o mau gerenciamento de estado é considerado a principal
causa de complexidade em sistemas contemporâneos, pois
impactam a compreensibilidade do programa, e os testes
de software.
Uma análise das aplicações \emph{desktop} da Adobe
apontou que a lógica de coordenação de eventos residia
em um terço do código e contia metade dos \emph{bugs}
reportados \cite{jarvi2008property}.
Interfaces gráficas inerentemente dispõem de um alto
grau de interativade, tornando seu desenvolvimento e
manutenção um desafio.

\emph{Programação Funcional Reativa (PFR)} é uma alternativa promissora para o
desenvolvimento de sistemas reativos, e tem sido explorada em vários domínios,
como: animação digital \cite{Elliott-H:1997:Fran}, GUIs \cite{Czaplicki:2012:Elm},
jogos digitais, robótica, e síntese de música.
PFR permite que aplicações interativas sejam programadas de forma declarativa
em um nível mais elevado de abstração, com código fonte que expressa melhor a
solução implementada.
Como resultado, o programa se torna mais compreensível, mais fácil de dar
manutenção e, em geral, menos complexo.

Este trabalho pretende contextualizar a situação atual
de como interfaces orientadas a eventos são implementadas,
apresentar os conceitos das abordagens alternativas, e
fornecer implementações de alguns componentes comuns em
interfaces gráficas, com a finalidade de testar e comparar
tais abordagens através do uso de algumas ferramentas
(linguagens e/ou \emph{frameworks}). Atenção especial será
dada ao ambiente \emph{web}, ou seja, interfaces de
aplicações utilizadas nos navegadores, e dependendo do
levantamento feito, alguns exemplos poderão ser dados
com alguma ferramenta \emph{FRP} para o ambiente
\emph{mobile} e/ou \emph{desktop}.

%%% Local Variables:
%%% mode: latex
%%% TeX-master: "../projeto"
%%% End:
