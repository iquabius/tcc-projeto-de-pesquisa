\section{Metodologia}\label{lmetodologia}

% Natureza: Aplicada
% Quanto aos Objetivos: Exploratória
% Pesquisa bibliográfica
% Quanto ao Procedimento: Estudo de Caso/casos de uso/casos de estudo
% Quanto a Abordagem: Qualitativa
% Coleta de dados: Observação direta

Apesar do paradigma \textit{FRP} ter sido apresentado há
quase duas décadas com o trabalho de \citeauthoronline{Elliott97franimation}
em \citeyear{Elliott97franimation}, sua exploração acadêmica
e aplicação por parte de pesquisadores e da indústria ainda é
recente. Por esse motivo este trabalho tem um objetivo de
cunho exploratório, que \citeauthoronline{gil2010metodos}
descreve da seguinte forma:

\begin{citacao}
  Pesquisas exploratórias são desenvolvidas com o
  objetivo de proporcionar visão geral, de tipo aproximativo,
  acerca de determinado fato.
  Este tipo de pesquisa é realizado especialmente quando o
  tema escolhido é pouco explorado e torna-se difícil sobre
  ele formular hipóteses precisas e operacionalizáveis
  (\citeyear{gil2010metodos}, p. 20).
\end{citacao}

A abordagem a ser utilizada será qualitativa, que não se
preocupa com a valores numéricos, mas procura aprofundar
a compreensão do objeto de estudo \cite[p.~31]{gerhardt2009metodos}.
A análise qualitativa será feita através de estudos de casos
implementados com as ferramentas a serem estudadas.
Para \citeauthoronline{santos2005manual}, um estudo de caso:

\begin{citacao}
  É o estudo que analisa com profundidade um ou poucos fatos,
  com vistas à obtenção de um grande conhecimento com riqueza
  de detalhes do objeto estudado. É usada nos estudos exploratórios
  e no início de pesquisas mas complexas. Tem aplicação em
  qualquer área do conhecimento (\citeyear{gil2010metodos}, p. 174).
\end{citacao}
