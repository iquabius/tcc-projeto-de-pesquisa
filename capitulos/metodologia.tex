\section{Metodologia}\label{lmetodologia}


Apesar do paradigma \textit{FRP} ter sido apresentado há
quase duas décadas com o artigo de \citeauthoronline{Elliott97franimation}
em \citeyear{Elliott97franimation}, sua exploração e aplicação
prática na indústria ainda é recente.
Por esse motivo este trabalho tem um objetivo de
cunho exploratório, que \citeauthoronline{gil2010metodos}
descreve da seguinte forma:

\begin{citacao}
  Pesquisas exploratórias são desenvolvidas com o
  objetivo de proporcionar visão geral, de tipo aproximativo,
  acerca de determinado fato.
  Este tipo de pesquisa é realizado especialmente quando o
  tema escolhido é pouco explorado e torna-se difícil sobre
  ele formular hipóteses precisas e operacionalizáveis
  (\citeyear{gil2010metodos}, p. 20).
\end{citacao}

Esse trabalho pretende estudar a usabilidade dos recursos
oferecidos pelos \emph{frameworks}/linguagens para a
programação de interfaces gráficas.
Como as propriedades a serem avaliadas não são quantificáveis,
a abordagem a ser utilizada será \emph{qualitativa}.
Uma avaliação qualitativa não se
preocupa com valores numéricos, mas procura aprofundar
a compreensão do objeto de estudo \cite[p.~31]{gerhardt2009metodos}.
A análise qualitativa será feita no código fonte de estudos de caso
implementados com as ferramentas a serem apresentadas
-- o código utilizado para implementar as interfaces será
avaliado, e não as interfaces em si.
Para \citeauthoronline{santos2005manual}, um estudo de caso:

\begin{citacao}
  É o estudo que analisa com profundidade um ou poucos fatos,
  com vistas à obtenção de um grande conhecimento com riqueza
  de detalhes do objeto estudado. É usada nos estudos exploratórios
  e no início de pesquisas mas complexas. Tem aplicação em
  qualquer área do conhecimento (\citeyear{gil2010metodos}, p. 172).
\end{citacao}
