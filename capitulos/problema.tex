\section{Formulação do problema}\label{lproblema}

\subsection{Variáveis}

\begin{itemize}[noitemsep]
  \item Aplicações web interativas
  \item \emph{Programação Assíncrona}
  \item Coordenação de eventos
    \begin{itemize}[noitemsep]
      \item Internos: \emph{mouse}, teclado, \emph{touchscreen}, etc.
      \item Externos: mensagens de servidores remotos.
    \end{itemize}
  \item Programação Funcional Reativa \emph{(FRP - Functional Reactive Programming)}
\end{itemize}


\subsection{Pergunta}

O quê o paradigma \emph{FRP}
pode oferecer para lidar com a complexidade de se coordenar
eventos assíncronos em aplicações web interativas?

% Variáveis
% - produtividade no desenvolvimento
% - coesão da base de código
% - expressividade do código
% - composicionalidade (compositionality)
% - previsibilidade do sistema
% - testabilidade
% - easier to reason about
% - testes de software
% - raciocínio informal (informal reasoning)
% - efeitos colaterais (side effects)
