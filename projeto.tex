\documentclass[
  12pt,         % tamanho da fonte
  a4paper,      % tamanho do papel.
  oneside,
  chapter=TITLE,      % títulos de capítulos convertidos em letras maiúsculas
  section=TITLE,
  english,      % idioma adicional para hifenização
  brazil,       % o último idioma é o principal do documento
  ]{abntex2}

% ---
% PACOTES
% ---

% ---
% Pacotes fundamentais
% ---
\usepackage{lmodern}      % Usa a fonte Latin Modern
\usepackage[T1]{fontenc}    % Selecao de codigos de fonte.
\usepackage[utf8]{inputenc}    % Codificacao do documento (conversão automática dos acentos)
\usepackage{indentfirst}    % Indenta o primeiro parágrafo de cada seção.
\usepackage{color}        % Controle das cores
\usepackage{graphicx}      % Inclusão de gráficos
\usepackage{microtype}       % para melhorias de justificação
% \usepackage{enumitem}
% \usepackage[normalem]{ulem}
% \usepackage{makeidx}
% \usepackage{amsmath}
\usepackage{unemat}

% pacotes relacionados a tabelas
\usepackage{amsfonts} % checkmark symbol
\usepackage{multirow} % expande uma célula pra várias linhas
\usepackage{diagbox}  % usado para desenhar um linha diagonal numa célula
\usepackage[export]{adjustbox} % centraliza tabelas/figuras além das margens do texto
% ---

% ---
% Pacotes de citações
% ---
\usepackage[brazilian,hyperpageref]{backref}
% "num" = numéricas; alf = autor-data
\usepackage[alf,abnt-etal-cite=2,abnt-etal-list=0]{abntex2cite}
\usepackage{csquotes}

% ---
% CONFIGURAÇÕES DE PACOTES
% ---

% ---
% Configurações do pacote backref
% Usado sem a opção hyperpageref de backref
\renewcommand{\backrefpagesname}{Citado na(s) página(s):~}
% Texto padrão antes do número das páginas
\renewcommand{\backref}{}
% Define os textos da citação
\renewcommand*{\backrefalt}[4]{
  \ifcase #1 %
    Nenhuma citação no texto.%
  \or
    Citado na página #2.%
  \else
    Citado #1 vezes nas páginas #2.%
  \fi}%
% ---

% ---
% Informações de dados para CAPA e FOLHA DE ROSTO
% ---
\titulo{Coordenação de Eventos na Programação de
        Interfaces Gráficas do Usuário com
        Programação Funcional Reativa}

\autor{Josias Duarte Busiquia}
\local{Barra do Bugres -- MT}
\data{2017}

\instituicao{%
  Universidade do Estado de Mato Grosso - UNEMAT
  \par
  Faculdade de Ciências Exatas e Tecnológicas - FACET
  \par
  Departamento de Ciência da Computação}

\tipotrabalho{Monografia}

% O preambulo deve conter o tipo do trabalho, o objetivo,
% o nome da instituição e a área de concentração
\preambulo{Projeto de pesquisa apresentado ao curso de Ciência da Computação da
  Universidade do Estado de Mato Grosso, como requisito parcial para obtenção do
  grau de Bacharel, sob orientação do Professor M.Sc. Alexandre Berndt.}
% ---

% ---
% Configurações de aparência do PDF final

% alterando o aspecto da cor azul
\definecolor{blue}{RGB}{41,5,195}

% informações do PDF
\makeatletter
\hypersetup{
  %pagebackref=true,
  pdftitle={\@title},
  pdfauthor={\@author},
  pdfsubject={\imprimirpreambulo},
  pdfcreator={LaTeX with abnTeX2},
  pdfkeywords={abnt}{latex}{abntex}{abntex2}{projeto de pesquisa},
  colorlinks=true,           % false: boxed links; true: colored links
  linkcolor=black,            % color of internal links
  citecolor=black,            % color of links to bibliography
  filecolor=black,          % color of file links
  urlcolor=black,
  bookmarksdepth=4
}
\makeatother
% ---

% ---
% Espaçamentos entre linhas e parágrafos
% ---

% O tamanho do parágrafo é dado por:
\setlength{\parindent}{1.3cm}

% Controle do espaçamento entre um parágrafo e outro:
\setlength{\parskip}{0.2cm}  % tente também \onelineskip

\makeindex

\begin{document}

% Seleciona o idioma do documento (conforme pacotes do babel)
\selectlanguage{brazil}

% Retira espaço extra obsoleto entre as frases.
\frenchspacing

% ----------------------------------------------------------
% ELEMENTOS PRÉ-TEXTUAIS (Preamble)
% ----------------------------------------------------------
% \pretextual

\imprimircapa
\imprimirfolhaderosto
% ---

% ---
% inserir lista de ilustrações
% ---
%\pdfbookmark[0]{\listfigurename}{lof}
%\listoffigures*
%\cleardoublepage
% ---

% ---
% inserir lista de tabelas
% ---
%\pdfbookmark[0]{\listtablename}{lot}
%\listoftables*
%\cleardoublepage
% ---

% ---
% inserir lista de abreviaturas e siglas
% ---
\begin{siglas}
  \item[FRP] \emph{Functional Reactive Programming}
  \item[HTML5] \emph{HyperText Markup Language}, versão 5
\end{siglas}


% ---
% inserir o sumario
% ---
\pdfbookmark[0]{\contentsname}{toc}
{\center\tableofcontents*}
\cleardoublepage
% ---

% ----------------------------------------------------------
% ELEMENTOS TEXTUAIS
% ----------------------------------------------------------
\textual

% ----------------------------------------------------------
% Capitulo de textual
% ----------------------------------------------------------

% \include força uma quebra de página
% \chapter{Projeto de Pesquisa}
\chapter*{Projeto de Pesquisa}
\markboth{Projeto de Pesquisa}{Projeto de Pesquisa}
\addcontentsline{toc}{chapter}{Projeto de Pesquisa}
\section{Tema}\label{ltema}

Programação de aplicações orientadas a eventos.

\section{Delimitação do tema}\label{ldelimitacao}

Interfaces \textit{Web} com Programação Funcional Reativa

\section{Problema}\label{lproblema}

O quê a Programação Funcional Reativa (PFR) tem a oferecer, e como ela se
compara aos modelos baseados em \emph{callbacks} quanto ao nível de abstração
fornecido para coordenação de eventos na programação de GUIs?

% The observer pattern violates an impressive line-up of important
% software engineering principles, like "side-effects, encapsulation,
% composibility, separation of concerns, scalability, uniformity,
% abstraction, resource managment, and semantic distance.
% ~ Maier (2010), Deprecating the Observer Pattern
%
% "Unfortunately event-driven programming can create more coordination
% problems than it solves." ~ Edwards (2009), Coherent Reaction
%
% "Unfortunately, programming with events comes at a cost:
% event-driven programs are extremely difficult to understand
% and maintain." ~ Fischer (2007), Tasks: Language Support for
% Event-driven Programming
%
% "This structure is a poor-man's concurrency: the event-handlers are
% coroutines and the event-loop is the scheduler." ~ Reppy (1992),
% Higher-order concurrency

%%% Local Variables:
%%% mode: latex
%%% TeX-master: "../projeto"
%%% End:

\section{Hipóteses}\label{lhipoteses}


\subsection{Hipótese Básica}

A interatividade intrínseca de interfaces gráficas de
aplicações web lhes dão a característica de serem altamente
orientadas a eventos,
ou seja, precisam reagir a vários comandos do usuário e/ou
mensagens de servidores remotos.
Atualmente o paradigma de \emph{Programação Orientada a
Eventos (EDP -- Event-driven Programming)} é o mais utilizado em
aplicações interativas, no entando muitos o consideram uma
das formas mais complexas para se desenvolver tais aplicações.

Recentemente o paradigma \emph{FRP} tem sido explorado como
uma alternativa promissora para a programação de aplicações
interativas, devido a maneira declarativa que é oferecida
para se expressar código de coordenação de eventos,
resultando em melhorias nos processos de desenvolvimento,
manutenção e testes de softwares.


\subsection{Hipóteses Secundárias}

\begin{itemize}[noitemsep]
  \item \emph{FRP} fornece um modelo de programação com um
        nível mais elevado de abstração.
  \item A expressividade de um sistema \emph{FRP} pode
        realçar a legibilidade do código.
  \item O paradigma \emph{FRP} pode ser difícil de ser adotado
        devido ao alto nível de abstração.
  \item O processo de \emph{debugging} pode ser mais difícil,
        devido a falta de ferramentas adequadas.
\end{itemize}

% Hipóteses
% - PF torna um sistema mais previsível através do gerenciamento do estado
% - PF oferece melhor reuso de código através da composicionalidade

\section{Objetivos}
\label{sec:objetivos}

\subsection{Objetivo Geral}

Demonstrar e analisar a PFR em comparação a modelos baseados em
\emph{callbacks}, quanto ao nível de abstração fornecido à coordenação de
eventos na programação de GUIs.

\subsection{Objetivos Específicos}

\begin{itemize}[noitemsep]
  \item Demonstrar o modelo \emph{declarativo} da \emph{Programação Funcional},
    e o modelo tradicional \emph{imperativo}, aplicados na manipulação de
    sequências (e.g. \emph{arrays}, listas, mapas);
  \item Demonstrar o modelo \emph{declarativo} da PFR, e o modelo tradicional
    \emph{imperativo} baseado \emph{callbacks}, aplicados na coordenação de
    eventos em GUIs;
  \item Analisar o modelo \emph{declarativo} em contraste com o
    \emph{imperativo} no que concerne o nível de abstração fornecido à
    manipulação de sequências, e à coordenação de eventos;
\end{itemize}

%%% Local Variables:
%%% mode: latex
%%% TeX-master: "../projeto"
%%% End:

\section{Justificativa}\label{ljustificativa}

% - websockets, push notifications, AppCache, service workers, web workers
%   - http://bit.ly/serviceworkers_webworkers_websockets
Interatividade em páginas \textit{web} se deu com a
introdução do \textit{JavaScript} em navegadores.
O advento de outras tecnologias tem tornado as interfaces
\textit{web} cada vez mais interativas
(e.g. Ajax e \textit{Web Sockets}\footnote{
  Tecnologia recentemente definida pela especificação do HTML5.
}),
dando origem a uma nova gama de aplicações \textit{web}
com interfaces ricas que oferecem ao usuário uma experiência
similar as aplicações \textit{mobile} ou \textit{desktop}.
Assim como em qualquer interface gráfica, interfaces \textit{web}
precisam reagir a vários eventos imprevisíveis do ambiente
externo, provindos tanto do usuário (e.g. \textit{clicks}
do \textit{mouse}, pressionamento de teclas, etc)
quanto de outro software (e.g. mensagens do servidor).

%  in a timely fashion = em tempo hábil
Atualmente o modelo de programação mais empregado na
coordenação desses eventos em programas interativos é o
\textit{event-driven programming}\footnote{
  Programação orientada a eventos.
},
que consiste de um \textit{event-loop} que espera por
eventos de forma contínua, e quando um evento é detectado,
uma função de \textit{callback} apropriada é chamada para
tratá-lo.
Essa abordagem configura uma das formas mais complexas de se
programar sistemas interativos \cite{
  edwards2009coherent,
  maier2010deprecating,
  reppy1992higher},
devido ao fato de que aplicações desenvolvidas utilizando
esse mecanismo apresentam um fluxo de controle desestruturado
% implícito (Flapjax)
e imprevisível, além de depender crucialmente de
\textit{efeitos colaterais\footnotemark} pra
\footnotetext{
  Do inglês \textit{side-effects:} característica muito comum
  em linguagens imperativas, onde uma função ou expressão pode
  modificar algum estado externo (e.g. alterar uma variável
  global, produzir uma saída na tela/terminal, escrever no
  sistema de arquivos, etc). Em programação funcional o uso de
  efeitos colaterais é desencorajado, e deve ser usado apenas
  quando absolutamente necessário -- e.g. manipular uma variável
  global não é absolutamente necessário, mas imprimir uma mensagem
  na tela pode ser.
}
gerenciar seu estado \cite{
  meyerovich2009flapjax,
  muller2015interactive,
  muller2015practical}.
Na literatura, essa abordagem é descrita como \textit{"Callback Hell"},
devido a forma desconcertante com que o fluxo de controle coordena
mudanças no estado do programa \cite[p.~2]{edwards2009coherent}.
% bainomugisha2013survey, muller2015practical
% inversão de controle

Vale ressaltar que a preocupação desnecessária com o
fluxo de controle e o mau gerenciamento de estado são
consideradas as principais causas de complexidade em
sistemas contemporâneos, pois afetam o entendimento das
várias partes do código por parte do desenvolvedor,
além de dificultar a realização de testes de software
\cite{Moseley06outof}.
Uma análise das aplicações \textit{desktop} da Adobe,
relatada em 2006, indicou que o código que coordena a
lógica de manuseio de eventos, \textit{widgets}, e outros
componentes da interface gráfica, representa cerca de
um terço do código, e mais da metade dos \textit{bugs}
reportados \cite{jarvi2008property}.
Sendo interfaces gráficas com alto grau de interatividade
parte inerente de uma aplicação, seu desenvolvimento e
manutenção se tornam um desafio.

%   - observer pattern

% declarative vs imperative
%   - specification (what) vs. execution (how)
%     - Declarative Interaction Design for Data Visualization
%   - modeling vs presentation
%     - Elm
%     - FR Animation
%
% FRP
%  - Outros tipos de software podem ser considerados reativos, como um sistema
%    embarcado que reage a sinais de sensores, ou um sistema distribuído que
%    precisa reagir a mensagens na rede.

<Apresentar FRP como alternativa>

% - Documentar o estado da arte em:
%   - técnologias web
%   - programação assíncrona
% "Este trabalho tem por objetivo apresentar os conceitos, objetivos, tecnologias e
% demais questões envolvidas na abordagem de desenvolvimento de aplicações
% Web conhecida como Ajax. E prover uma aplicação Web de
% georeferenciamento do campus da UFSC utilizando a abordagem Ajax."

<Descrever proposta do projeto>

\section{Fundamentação Teórica}
\label{sec:fund-teor}

\subsection{Interfaces Gráficas do Usuário (GUIs)}
\label{sec:guis}

\subsubsection{GUIs em Aplicações Desktop}
\label{sec:guis-desktop}

% - utilizam toolkits

\subsubsection{GUIs em Aplicações Web}
\label{sec:guis-web}

%\subsubsubsection{JavaScript}
%\label{sec:javascript}
%
%\subsubsubsection{DOM \emph{(Document Object Model)}}
%\label{sec:dom}
%
%% \subsubsection{Programação com \emph{Callbacks}}

\subsection{Programação Funcional}
\label{sec:prog-funcional}

% "Functional Programming (enabled by lambdas with closure)"
% Contextualização Histórica
% Renascença da Programação Funcional


\subsubsection{Funções Puras \emph{(Pure Functions)}}
\label{sec:func-puras}

\subsubsubsection{Efeitos Colaterais \emph{(Side-effects)}}

\subsubsection{Funções de Primeira Classe \emph{(First Class Functions)}}

% - Expressões lambda / funções anônimas / Closures

\begin{listing}[H]
  \centering
  \caption{Atribuição de funções a variáveis}
  \begin{minted}
    [
    frame=lines,
    framesep=2mm,
    baselinestretch=1.2,
    fontsize=\scriptsize, % scriptsize, footnotesize, small
    linenos,
    mathescape
    ]
    {js}
    var sumar = function(x, y) {
      return x + y
    }

    // Utilizando 'let' e 'Arrow Functions' do JavaScript ES6
    let subtrair = (x, y) => {
      return x - y
    }

    // Quando a função possui apenas uma expressão
    // as chaves podem ser removidas
    let multiplicar = (x, y) => return x * y

    // Quando a única expressão é um retorno, a palavra chave
    // 'return' também pode ser removida
    let dividir = (x, y) => x / y
  \end{minted}
  \label{code:app-init}
\end{listing}

\begin{listing}[H]
  \centering
  \caption{Expressões \emph{lambda}}
  \begin{minted}
    [
    frame=lines,
    framesep=2mm,
    baselinestretch=1.2,
    fontsize=\scriptsize, % scriptsize, footnotesize, small
    linenos,
    mathescape
    ]
    {js}
    // A função 'f' chama a função 'g' passando o valor do parâmetro 'x'
    // como argumento.
    let f = (g, x) => g(x)

    // Dois valores são passados a função 'f':
    // O primeiro é uma função 'lambda' que retorna 'y + 1'
    // O segundo é o numero '99'
    let resultado = f((y) => y + 1, 99)
    // Ao executar a função 'f', a expressão lambda é vinculada
    // a variável 'g', definida na lista de argumentos.
  \end{minted}
  \label{code:app-init}
\end{listing}

\subsubsection{Funções de Ordem Superior \emph{(Higher-order Functions)}}

\emph{Funções de Ordem Superior} são funções que aceitam outras funções como
argumento, ou retornam uma função.

\subsubsubsection{Funções como argumentos}

\subsubsubsection{Funções como valores retornados}

\subsubsection{Primitivas Básicas da PF}
\label{sec:primitivas-pf}

% \emph{map}, \emph{filter}, \emph{fold}, \emph{reduce}, \emph{scan}, \emph{zip}.

\begin{listing}[H]
  \centering
  \caption{Primitiva \emph{map}}
  \begin{minted}
    [
    frame=lines,
    framesep=2mm,
    baselinestretch=1.2,
    fontsize=\scriptsize, % scriptsize, footnotesize, small
    linenos,
    mathescape
    ]
    {js}
    let numbers = [0, 1, 2, 3, 4, 5, 6, 7, 8, 9]

    // Computa uma lista com os quadrados dos valores de 'numbers'
    let squaredNumbers = numbers.map(num => Math.pow(num, 2))

    //-> [0, 1, 4, 9, 16, 25, 36, 49, 64, 81]
  \end{minted}
  \label{code:app-init}
\end{listing}


\subsubsection{Composição de Funções}

%\subsubsection{\emph{Currying}}
%

\subsection{Programação Orientada a Objetos e o \emph{Observer Pattern}}


\subsection{Programação Funcional Reativa (PFR)}
\label{sec:pfr}

% "FRP permits the modeling of systems that must respond to input over time in a
% simple and declarative manner." ~ Amsden (2011), Survey on FRP

% "A program in an FRP language generally corresponds quite closely to a
% mathematical model of the system being implemented." ~ Amsden (2011), Survey
% on FRP
%   - Programação Reativa
%     - “[…] is programming with asynchronous data streams” – André Staltz
%   - merge, replay, retry, skip, start, startWith


\subsubsection{As 10 Primitivas Básicas da PFR}
\label{sec:pfr-10-primitivas}

\emph{map}, \emph{merge}, \emph{hold}, \emph{snapshot}, \emph{filter},
\emph{lift}, \emph{never}, \emph{constant}, \emph{sample}, \emph{switch}.


\subsubsection{Arcabouços \emph{(Frameworks)} PFR}
\label{sec:pfr-frameworks}

Rx.JS, Bacon.js
% Ferramentas
%   Bibliotecas & Frameworks
%   Bacon.js
%   Cycle.js → Model-View-Intent
%   Elm → Model-Update-View
%   Rx
%   Meteor

%%% Local Variables:
%%% mode: latex
%%% TeX-master: "../projeto"
%%% End:

\section{Metodologia}\label{lmetodologia}


Apesar do paradigma \textit{FRP} ter sido apresentado há
quase duas décadas com o artigo de \citeauthoronline{Elliott97franimation}
em \citeyear{Elliott97franimation}, sua exploração e aplicação
prática na indústria ainda é recente.
Por esse motivo este trabalho tem um objetivo de
cunho exploratório, que \citeauthoronline{gil2010metodos}
descreve da seguinte forma:

\begin{citacao}
  Pesquisas exploratórias são desenvolvidas com o
  objetivo de proporcionar visão geral, de tipo aproximativo,
  acerca de determinado fato.
  Este tipo de pesquisa é realizado especialmente quando o
  tema escolhido é pouco explorado e torna-se difícil sobre
  ele formular hipóteses precisas e operacionalizáveis
  (\citeyear{gil2010metodos}, p. 20).
\end{citacao}

Esse trabalho pretende estudar a usabilidade dos recursos
oferecidos pelos \emph{frameworks}/linguagens para a
programação de interfaces gráficas.
Como as propriedades a serem avaliadas não são quantificáveis,
a abordagem a ser utilizada será \emph{qualitativa}.
Uma avaliação qualitativa não se
preocupa com valores numéricos, mas procura aprofundar
a compreensão do objeto de estudo \cite[p.~31]{gerhardt2009metodos}.
A análise qualitativa será feita no código fonte de estudos de caso
implementados com as ferramentas a serem apresentadas
-- o código utilizado para implementar as interfaces será
avaliado, e não as interfaces em si.
Para \citeauthoronline{santos2005manual}, um estudo de caso:

\begin{citacao}
  É o estudo que analisa com profundidade um ou poucos fatos,
  com vistas à obtenção de um grande conhecimento com riqueza
  de detalhes do objeto estudado. É usada nos estudos exploratórios
  e no início de pesquisas mas complexas. Tem aplicação em
  qualquer área do conhecimento (\citeyear{gil2010metodos}, p. 172).
\end{citacao}

\section{Cronograma}\label{lcronograma}

\begin{adjustbox}{center}
  \tiny
  \begin{tabular}{|p{4cm}|c|c|c|c|c|c|c|c|c|c|c|l|}
    \cline{1-12}
      \multicolumn{1}{|c|}{
        \multirow{2}{*}{
          \diagbox[width=4.4cm]{
            \textbf{Atividades}
          }{
            \textbf{Ano/Mês}
          }}}
      & \multicolumn{6}{c|}{\textbf{2016/1}}
      & \multicolumn{5}{c|}{\textbf{2016/2}} \\
      \cline{2-12}
      & Fev. & Mar. & Abr. & Maio & Jun. & Jul. & Ago. & Set. & Out. & Nov. & Dez. \\
    \hline
      Desenvolvimento do Tema e Objetivos
      & X &  &  &  &  &  &  &  &  &  &  \\
    \hline
      Desenvolvimento do Problema, Hipótese e Justificativa
      & & X & X & X &  &  &  &  &  &  &  \\
    \hline
      Desenvolvimento da Metodologia e Fundamentação Teórica
      & & & & X & X & X & & & & & \\
    \hline
      Desenvolvimento do Cronograma e Orçamento
      & & & X & & & & & & & & \\
    \hline
      Encontros de Orientação
      & & X & X & X & X & X & X & X & X & X & X \\
    \hline
      Defesa do Projeto de Pesquisa
      & & & & & X & & & & & & \\
    \hline
      Desenvolvimento da Introdução
      & & & & & & X & X & & & & \\
    \hline
      Desenvolvimento do TCC
      & & & & & & X & X & X & X & X & \\
    \hline
      Desenvolvimento dos Casos de Uso
      & & & & & & & X & X & X & & \\
    \hline
      Correção de Erros
      & & X & X & X & X & X & X & X & X & X & X \\
    \hline
      Elaboração da apresentação do TCC
      & & & & & & & & & & X & X \\
    \hline
      Apresentação do TCC
      & & & & & & & & & & & X \\
    \hline
      Entrega da Versão final
      & & & & & & & & & & & X \\
    \hline
  \end{tabular}
\end{adjustbox}

\section{Orçamento}\label{lorcamento}

\begin{center}
  \tiny
  \begin{tabular}{| l | l | l | l |}
  \hline
  \textbf{Descrição das Despesas} & \textbf{Quantidade} & \textbf{Valor Estimado (em reais)} \\ \hline
  Fotocópia & 200 & R\$ 100,00 \\ \hline
  Encadernação & 7 & R\$ 50 \\ \hline
  Impressão de Artigos & 20 & R\$ 75,00 \\ \hline
  1 ano de \textit{https://egghead.io} & 1 & R\$ 290,00 \\ \hline
  1 ano de ACM Digital Library & 1 & R\$ 107,00 \\ \hline
  Aquisição de bibliografias & 1 & R\$ 230,00 \\
  \hline \hline
  \textbf{Total} & & \textbf{R\$ 852,00} \\
  \hline
  \end{tabular}
\end{center}


% ---
% Finaliza a parte no bookmark do PDF
% para que se inicie o bookmark na raiz
% e adiciona espaço de parte no Sumário
% ---

\phantompart

% ----------------------------------------------------------
% ELEMENTOS PÓS-TEXTUAIS
% ----------------------------------------------------------
\postextual

% ----------------------------------------------------------
% Referências bibliográficas
% ----------------------------------------------------------
\bibliography{refs}

\phantompart
\printindex

\end{document}
